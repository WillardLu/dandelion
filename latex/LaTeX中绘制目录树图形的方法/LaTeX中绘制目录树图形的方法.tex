% 繁星间漫步,陆巍的博客
\documentclass[UTF8]{ctexart}

\usepackage[dvipsnames, svgnames, x11names]{xcolor}% 颜色支持
\usepackage{dirtree}% 绘制目录树
\usepackage[edges]{forest}% 绘制树状图形

\setlength{\parindent}{2em}% 缩进
\setlength{\parskip}{2ex} % 段间距

\begin{document}


\section{绘制目录树}


\subsection{示例一}


\subsubsection{使用dirtree宏包}
\dirtree{%
  .1 ..
  .2 \textcolor{blue}{my\_project}.
  .3 \textcolor{blue}{my\_project}.
}


\subsubsection{使用forest宏包}
\begin{forest}
  for tree={
    grow'=0,
    folder
  }
    [
      .
      [\textcolor{blue}{my\_project}
        [\textcolor{blue}{my\_project}]
      ]
    ]
\end{forest}


\subsection{示例二}


\subsubsection{使用dirtree宏包}
\dirtree{%
  .1 ..
  .2 \textcolor{blue}{my\_project}.
  .3 .gitignore.
  .3 LICENSE.
  .3 \textcolor{blue}{my\_project}.
  .4 \_\_init\_\_.py.
  .3 README.md.
  .3 requirements.txt.
  .3 setup.py.
  .3 \textcolor{blue}{tests}.
  .4 test\_my\_project.py.
}


\subsubsection{使用forest宏包}
\begin{forest}
  for tree={
    grow'=0,
    folder
  }
    [
      .
      [\textcolor{blue}{my\_project}
        [.gitignore]
        [LICENSE]
        [\textcolor{blue}{my\_project}
          [\_\_init\_\_.py]
        ]
        [README.md]
        [requirements.txt]
        [setup.py]
        [\textcolor{blue}{tests}
          [test\_my\_project.py]
        ]
      ]
    ]
\end{forest}

\end{document}
