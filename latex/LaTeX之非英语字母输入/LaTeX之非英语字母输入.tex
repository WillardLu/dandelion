% 繁星间漫步,陆巍的博客
\documentclass[UTF8, fontset=adobe]{ctexart}

\usepackage{geometry}% 用于页面设置
\usepackage{textgreek}% 希腊字母支持
\usepackage{cjhebrew}% 希伯来文支持

% 设置为A4纸
\geometry{
  a4paper,
  left = 1in,
  right = 1in,
  top = 1in,
  bottom = 1in,
  heightrounded,
}

% ------------------ 开始 -------------------

\begin{document}

\section{法语字母}
这里主要是指如何输入变音符号,下面是示例。
\begin{itemize}
  \item 闭音符:\'e 或者 {\'e},probl{\`e}me。
  \item 开音符:\`e 或者 {\`e},{\`e}tat。注意,这个是反单引号。
  \item 长音符:\^e 或者 {\^e},{\^a}me。
  \item 分音符:\"e 或者 {\"e},G{\"u}nter。
  \item 软音符:\c c 或者 {\c c},le{\c c}on。
\end{itemize}


\section{希腊字母}
默认情况下输入希腊字母,需要在数学环境下,例如\verb|$\alpha$|,如果不想在数学环境下,就使用textgreek宏包。

\vspace{2ex}
\begin{tabular}{p{2em}|p{2em}|p{2em}|p{2em}|p{2em}|p{2em}|p{2em}|p{2em}}
  \hline
  大写 & 小写 & 大写 & 小写 & 大写 & 小写 & 大写 & 小写\\
  \hline
  A & \textalpha & B & \textbeta & \textGamma & \textgamma & \textDelta & \textdelta\\
  \hline
  E & \textepsilon & Z & \textzeta & H & \texteta & \textTheta & \texttheta\\
  \hline
  I & \textiota & K & \textkappa & \textLambda & \textlambda & M & \textmu\\
  \hline
  N & \textnu & \textXi & \textxi & O & o & \textPi & \textpi\\
  \hline
  P & \textrho & \textSigma & \textsigma & T & \texttau & \textUpsilon & \textupsilon\\
  \hline
  \textPhi & \textphi & X & \textchi & \textPsi & \textpsi & \textOmega & \textomega\\
  \hline
\end{tabular}


\section{希伯来字母}


\subsection{辅音}
\huge
\<' b g d h w z .h .t y>

\<k K l m M n N s ` p>

\<P .s .S q r /s ,s +s t>


\subsection{元音、重音}
\<i e E E: a /a a: A A: o u * : O/wo U/w* ; -- \dottedcircle>
 
\end{document}
