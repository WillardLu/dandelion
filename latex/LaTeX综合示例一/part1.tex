part{第一卷}


\chapter{引言}
自古以来,关于连续地变化、生长和运动的直观概念,一直在向科学的见解挑战。但是,直到十七世纪,当现代科学同微分学和积分学(简称为微积分)以及数学分析密切相关地产生并迅速发展起来的时候,才开辟了理解连续变化的道路。

……

\ExecuteMetaData[figure_lib.tex]{figure1.1}

……

在$n=2$的特殊情况下,我们选取
$$
a_1=\sqrt{x},\ a_2=\sqrt{y},\ b_1=\sqrt{y},\ b_2=\sqrt{x},
$$
这里$x$和$y$都是正数。这时,柯西-希瓦兹不等式成为:$(2\sqrt{xy})^2\leqslant(x+y)^2$,或者
$$
\sqrt{xy}\leqslant\frac{x+y}{2}.
$$
此不等式表明:两个正数$x,y$的几何平均值$\sqrt{xy}$决不超过其\CJKunderdot{算术平均值}$\frac{x+y}{2}$。如果直角三角形的高将斜边分为两个线段,其长度分别为$x$和$y$,则两个数$x,y$的几何平均值就可解释为此高的长度。因此,上述不等式表明,在直角三角形中,斜边上的高不超过斜边的二分之一(见图1.6)\footnote{有兴趣的读者,可在下列著作中找到更多的资料:F. F. Beckenbach and R. Bellman,An Introduction to lnequalities(不等式引论),Random House,1961以及 N. Karzarinoff,Geometric Inequalities(几何不等式),Random House,1961。}。

\ExecuteMetaData[figure_lib.tex]{figure1.6}

……

在本章和以后各章中,我们几乎完全是研究单个自变量(譬如说$x$)和单个因变量(譬如说$y$)的情况,正如在例b中所表明的那样\footnote{然而,从一开始就应着重指出:在许多场合多变量函数的出现是很自然的。在第二卷中,将系统地讨论多变量函数。}。这种函数,我们通常是按标准方式,用它在$x,y$平面上的图形,即用由点$(x,y)$组成的曲线来表示的,曲线各点的纵坐标$y$同横坐标$x$满足特定的函数关系(见图1.7)。对于例b来说,其图形是围绕着坐标原点半径为1的圆的上半部。

\ExecuteMetaData[figure_lib.tex]{figure1.7}

另外,如把函数解释为由$x$轴上的定义域到$y$轴上的值域的映射,还可得到函数的另一种形象描述。这里我们不是把$x$和$y$解释为$x,y$平面上同一点的坐标,而是解释为两个不同的独立的数轴上的点。于是,函数就把$x$轴上的点$x$映射为$y$轴上的点$y$。这种映射在几何学中是常常会出现的,例如,把$x$轴上的点$x$投影到平行的$y$轴上的点$y$(投影中心0处于两轴所在平面内)时所产生的“仿射”映射(见图1.8)。不难断定,这一映射可用线性函数$y=ax+b$(其中$a$和$b$均为常数)解析地表示。显然,仿射映射是“一对一”的映射,其中每一个映象$y$反过来又对应着唯一的原象$x$。另一个更为一般的映射是由同一类投影定义的“透视映射”,只是,两轴不一定平行。其中,解析表达式由形如$y=(ax+b)/(cx+d)$的有理线性函数给出,其中$a,b,c,d$均为常数。

\ExecuteMetaData[figure_lib.tex]{figure1.8}
