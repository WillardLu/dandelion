% CSDN,繁星间漫步,陆巍的博客
\documentclass[UTF8,a4paper]{ctexart}

% 设置章节标题左对齐,+=表示在原有格式上追加,如果只有=则表示完全替换
\ctexset{
  section/format += \raggedright,
  subsection/format += \raggedright,
  subsubsection/format += \raggedright,
}

\setlength{\parindent}{2em}% 缩进
\setlength{\parskip}{2ex} % 段间距


% ------------------ 开始 -------------------
\begin{document}
\section{俄语字母的显示}
1、没有使用$\backslash$font与$\backslash$rm命令的情况下

这一尝试,使得他和他的学派为此苦苦奋斗二十来年,在20世纪初发现的贝纳德(Benard)对流激起了它们的灵感,20世纪50年代贝洛索夫(Beдycoв)和扎鲍廷斯基(Жaбoинcкий)等人所作的化学振荡实验推动了他们的理论研究,最后终于得到了“耗散结构”概念,并于1967年在一次国际学术会议上公布于世。

2、使用了$\backslash$font与$\backslash$rm命令的情况

这一尝试,使得他和他的学派为此苦苦奋斗二十来年,在20世纪初发现的贝纳德(Benard)对流激起了它们的灵感,20世纪50年代贝洛索夫(\font\rm="Ubuntu Mono"\rm Beдycoв)和扎鲍廷斯基(Жaбoинcкий)等人所作的化学振荡实验推动了他们的理论研究,最后终于得到了“耗散结构”概念,并于1967年在一次国际学术会议上公布于世。

3、前面使用$\backslash$font与$\backslash$rm命令对后面字符显示的影响

这一尝试,使得他和他的学派为此苦苦奋斗二十来年,在20世纪初发现的贝纳德(Benard)对流激起了它们的灵感,20世纪50年代贝洛索夫(Beдycoв)和扎鲍廷斯基(Жaбoинcкий)等人所作的化学振荡实验推动了他们的理论研究,最后终于得到了“耗散结构”概念,并于1967年在一次国际学术会议上公布于世。

4、在$\backslash$font与$\backslash$rm命令后设置字体大小

这一尝试,使得他和他的学派为此苦苦奋斗二十来年,在20世纪初发现的贝纳德(Benard)对流激起了它们的灵感,20世纪50年代贝洛索夫(\font\rm="Ubuntu Mono" at 18pt\rm Beдycoв)和扎鲍廷斯基(Жaбoинcкий)等人所作的化学振荡实验推动了他们的理论研究,最后终于得到了“耗散结构”概念,并于1967年在一次国际学术会议上公布于世。\normalsize

5、前面使用$\backslash$normalsize 命令后对字符显示的影响

这一尝试,使得他和他的学派为此苦苦奋斗二十来年,在20世纪初发现的贝纳德(Benard)对流激起了它们的灵感,20世纪50年代贝洛索夫(Beдycoв)和扎鲍廷斯基(Жaбoинcкий)等人所作的化学振荡实验推动了他们的理论研究,最后终于得到了“耗散结构”概念,并于1967年在一次国际学术会议上公布于世。

6、再次使用$\backslash$font与$\backslash$rm命令

这一尝试,使得他和他的学派为此苦苦奋斗二十来年,在20世纪初发现的贝纳德(Benard)对流激起了它们的灵感,20世纪50年代贝洛索夫(Beдycoв)和扎鲍廷斯基(\font\rm="Ubuntu Mono"\rm Жaбoинcкий)等人所作的化学振荡实验推动了他们的理论研究,最后终于得到了“耗散结构”概念,并于1967年在一次国际学术会议上公布于世。



\section{希腊字母的显示}
\normalsize

Αα、Ββ、Γγ、Δδ、Εε、Ϝϝ、Ζζ、Ηη、Θθ、Ιι、Κκ、Λλ、Μμ、Νν、Ξξ、Οο、Ππ、Ρρ、Σσ、Ττ、Υυ、Φφ、Χχ、Ψψ、Ωω

\font\rm="CMU Serif"\rm 

Αα、Ββ、Γγ、Δδ、Εε、Ϝϝ、Ζζ、Ηη、Θθ、Ιι、Κκ、Λλ、Μμ、Νν、Ξξ、Οο、Ππ、Ρρ、Σσ、Ττ、Υυ、Φφ、Χχ、Ψψ、Ωω

\end{document}